\documentclass[12pt]{article}

\usepackage{geometry}
\usepackage[utf8]{inputenc}
\usepackage[polish]{babel}
\usepackage{polski}
\usepackage{hyperref}
\usepackage{graphicx}
\usepackage{verbatim}
\usepackage{acronym}
\usepackage{fancyhdr}
\usepackage[usenames]{color}

\hypersetup{
  linkbordercolor={1 1 1},
  urlbordercolor={1 1 1},
  colorlinks=false
}

\newpage


\begin{document}

\section{Historie}

\subsection{Story 20}
\textbf{Tytul:} Rozpoczecie gry \\
\textbf{Opis:} wszystkim graczom przydzielana jest ustalona ilosc dublonow, odpowiedni
znacznik plantacji; ustalana jest ilosc statkow oraz ilosc dostepnych
kolonistow. \\
\textbf{Priorytet:} M \\
\textbf{Koszt:} 1h \\

\subsection{Story 18 }
\textbf{Tytul:} Budowa budynkow \\
\textbf{Opis:}  Budynki mają swoj koszt; Koszt budynkow obnizany moze byc przez
ilosc aktywnych kamieniolomow (max. wartosc obnizenia to nr grupy do ktorej
nalezy budynek);przy kazdym budynku znalezc sie musi cena, opis oraz ilosc
miejsc dla kolonistow\\
\textbf{Priorytet:} C \\
\textbf{Koszt:} 3h \\

\subsection{Story 17}
\textbf{Tytul:} Produkcja typu towaru \\
\textbf{Opis:} Produkowane moze byc tyle towaru, ile przetworzone moze byc przez
budynki (tyle towaru ile kolonistow w przetworniach, gdy jest tyle plantacji)\\
\textbf{Priorytet:} C \\
\textbf{Koszt:} 1h \\


\subsection{Story 19}
\textbf{Tytul:} Role specjalne budynków \\
\textbf{Opis:} Budynki mają funkcje które modyfikują zasady \\
\textbf{Priorytet:} W \\
\textbf{Koszt:} ?h \\

\subsection{Story }
\textbf{Tytul:}  \\
\textbf{Opis:}  \\
\textbf{Priorytet:}  \\
\textbf{Koszt:}  \\

\end{document}

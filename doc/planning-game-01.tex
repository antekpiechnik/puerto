\documentclass[12pt]{article}

\usepackage{geometry}
\usepackage[utf8]{inputenc}
\usepackage[polish]{babel}
\usepackage{polski}
\usepackage{hyperref}
\usepackage{graphicx}
\usepackage{verbatim}
\usepackage{acronym}
\usepackage{fancyhdr}
\usepackage[usenames]{color}

\hypersetup{
  linkbordercolor={1 1 1},
  urlbordercolor={1 1 1},
  colorlinks=false
}

\newpage

\renewcommand{\arraystretch}{1.5}

\newcommand{\story}[5]{
\begin{tabular*}{15cm}{@{\extracolsep{\fill}}| p{12cm} | c |}
\hline
\large{\textbf{#2}} & \makebox[2cm][c]{\large{Nr: \textbf{#1}}} \\
\hline
\multicolumn{2}{| p{14cm} |}{#5} \\
\hline
Priorytet: \textbf{#3} & Czas: \textbf{#4h} \\
\hline
\end{tabular*}
\vspace{1cm}
}

\begin{document}

\section{Historie}
Historie użytkownika prezentowane w odpowiednim formacie.
\vspace{1cm}

\story{1}{Ekran powitalny}{Must}{1}{Nowa gra; High scores; Exit}

\story{2}{Wybór graczy}{Must}{1}{3-5 graczy. Możliwość wpisania nazw.}

\story{3}{Ekran stanu gracza}{Must}{3}{W dowolnej chwili mogę włączyć i
podpatrzyć stan każdego gracza (poza PZ innych graczy). Na stan składają się:
\begin{itemize}
  \item budynki
  \item uprawy
  \item Punkty Zwycięstwa
  \item dublony
  \item surowce
\end{itemize}
}

\subsection{Story 4}
\textbf{Tytul:} Ekran stanu gry \\
\textbf{Opis:} Dostepne budynki;Kto jest gubernatorem;Jaka jest wybrana
faza;Status Targowiska;Ilosc dostepnych PZ;Ilosc dostepnych typow
surowcow;Status statkow \\
\textbf{Priorytet:} M \\
\textbf{Koszt:} 3h \\

\subsection{Story 20}
\textbf{Tytul:} Rozpoczecie gry \\
\textbf{Opis:} wszystkim graczom przydzielana jest ustalona ilosc dublonow, odpowiedni
znacznik plantacji; ustalana jest ilosc statkow oraz ilosc dostepnych
kolonistow. \\
\textbf{Priorytet:} M \\
\textbf{Koszt:} 1h \\

\subsection{Story 7}
\textbf{Tytul:} Zakonczenie gry \\
\textbf{Opis:} Gra konczy sie gdy: 1. Dowolny gracz zajal wszystkie miejsca w
miescie; 2. Po fazie burmistrza nie mozna dolozyc pracownikow do budynkow lub
upraw. 3. Po fazie kapitana skonczyly sie PZ. Gra konczy sie do konca danej tury
i potem jest konczona. \\
\textbf{Priorytet:} C \\
\textbf{Koszt:} 1h \\

\subsection{Story 16}
\textbf{Tytul:} Rola Kolonistow  \\
\textbf{Opis:} Budynek lub plantacja jest aktywna tylko gdy jest zamieszkiwana
  przez kolonistow \\
\textbf{Priorytet:} C \\
\textbf{Koszt:} 2h \\

\subsection{Story 18 }
\textbf{Tytul:} Budowa budynkow \\
\textbf{Opis:}  Budynki mają swoj koszt; Koszt budynkow obnizany moze byc przez
ilosc aktywnych kamieniolomow (max. wartosc obnizenia to nr grupy do ktorej
nalezy budynek);przy kazdym budynku znalezc sie musi cena, opis oraz ilosc
miejsc dla kolonistow\\
\textbf{Priorytet:} C \\
\textbf{Koszt:} 3h \\

\subsection{Story 17}
\textbf{Tytul:} Produkcja typu towaru \\
\textbf{Opis:} Produkowane moze byc tyle towaru, ile przetworzone moze byc przez
budynki (tyle towaru ile kolonistow w przetworniach, gdy jest tyle plantacji)\\
\textbf{Priorytet:} C \\
\textbf{Koszt:} 1h \\

\subsection{Story 5}
\textbf{Tytul:} Faza Gry \\
\textbf{Opis:} Wybieram jedna z dostepnych rol i kazdy gracz (poczawszy ode
mnie) wykonuje zwiazana czynnosc;Dla wybierajacego dostepna jest
gratyfikacja zwiazana z dana rola \\
\textbf{Priorytet:} S \\
\textbf{Koszt:} 2h \\

\subsection{Story 6}
\textbf{Tytul:} Tura Gry \\
\textbf{Opis:} Tyle faz ile graczy;Fazy wybieramy w ustalonej kolejnosci
poczawszy od gubernatora;Po skonczeniu gubernatorem zostaje nastepny gracz, a na
kazda niewykorzystana role kladzie sie po Dublonie z Banku;Na koniec sprawdzanie
warunkow zakonczenia gry\\
\textbf{Priorytet: S}  \\
\textbf{Koszt:} 1h \\

\subsection{Story 9}
\textbf{Tytul:} Rola - Poszukiwacz \\
\textbf{Opis:} Kto wybral dostaje 1 Dublon z banku. \\
\textbf{Priorytet:} S \\
\textbf{Koszt:} 0h \\

\subsection{Story 10}
\textbf{Tytul:} Rola - Budowniczy \\
\textbf{Opis:} Mozna wybrc budynek (z puli dostepnych) i go zbudowac;Wybierajacy
buduje 1 Dublon taniej\\
\textbf{Priorytet:} S \\
\textbf{Koszt:} 2h \\

\subsection{Story 11}
\textbf{Tytul:} Rola - Burmistrz \\
\textbf{Opis:} Poczawszy od burmistrza rozdaje sie po jednym koloniscie kazdemu
graczowi az sie nie skoncza; Gracze moga przemieszczac kolonistow; Liczy sie
ilosc wolnych miejsc w budynkach u wszystkich graczy i tyle kolonistow daje sie
na statek, ale min. ilosc graczy\\
\textbf{Priorytet:} S \\
\textbf{Koszt:} 4h \\

\subsection{Story 12 }
\textbf{Tytul:} Rola - Nadzorca \\
\textbf{Opis:} Poczawszy od nadzorcy gracze produkuja towar;W kazdej kolejce
tylko jeden typ produktu. Kolejki az sie nie skoczy mozliwosc tworzenia;Nazdorca
dostaje jeden znacznik typu ktory wyprodukowal wiecej\\
\textbf{Priorytet:} S \\
\textbf{Koszt:} 2h \\

\subsection{Story 13}
\textbf{Tytul:} Rola - Kupiec \\
\textbf{Opis:} Mozna sprzedawac towary na targowiskach;Kupiec dostaje 1 Dublon
wiecej za kazdy sprzedany towar; Sa 4 miejsca, tylko jeden znacznik danego typu
na targowisku;Sprzedaz po 1 towarze, trwa dopoki mozna i gracze chca;Jesli
zapelnily sie miejsca to targowisko sie czysci \\
\textbf{Priorytet:} S \\
\textbf{Koszt:} 2h \\

\subsection{Story 14}
\textbf{Tytul:} Rola - Plantator \\
\textbf{Opis:} Wybor plantacji sposrod dostepnych (zawsze 1 wiecej niz ilosc
graczy, losuje sie sposrod puli); zaczyna sie od plantatora; jego
przywilejem jest kamieniolom zamiast plantacji \\
\textbf{Priorytet:} S \\
\textbf{Koszt:} 2h \\

\subsection{Story 15 }
\textbf{Tytul:} Rola - Kapitan \\
\textbf{Opis:} Zaladunek - ladowanie po jednym typie towaru na statek;Za kazdy
znacznik gracz otrzymuje 1 PZ;Kapitan za pierwszym razem 1PZ wiecej;Na kazdym
statku jeden typ towaru, ilosc i rozmiar statkow ustalone z poczatkiem gry \\
\textbf{Priorytet: S}  \\
\textbf{Koszt: 3h}  \\

\subsection{Story 8}
\textbf{Tytul:} Okreslenie zwyciezcy \\
\textbf{Opis:} Wygrywa ten kto ma najwiecej PZ \\
\textbf{Priorytet:} W \\
\textbf{Koszt:} 0h \\

\subsection{Story 19}
\textbf{Tytul:} Role specjalne budynków \\
\textbf{Opis:} Budynki mają funkcje które modyfikują zasady \\
\textbf{Priorytet:} W \\
\textbf{Koszt:} ?h \\

\section{Zadania}

\end{document}
